\section{Create an USB Stick}

There are a few requirements to use a USB Stick:
\begin{itemize}
	\item The filesystem on the USB Stick must be \textbf{FAT32}
	\item There has to be a special directory structure on the stick
\end{itemize}

\subsection{Required directory structure}
There have to be several directories on the USB Stick the tools expect:

\dirtree{%
	.1 /.
	.2 openvario.
	.3 maps.
	.3 repo.
	.3 download.
	.4 xcsoar.
	.3 upload.
	.4 xcsoar.
	.3 images.
}

\subsubsection{maps}
This directory is used for updating map files. 
The format of  the map files can either be *.xcm or a compiled ipk for the opkg packet manager.
The map file will be copied/installed to XCSoar settings directory and can be selected using the XCSoar map settings dialog

\subsubsection{repo}
This directory is used for software updates of the OV-Linux system on a packet basis.
The whole repository directory on the FTP server has to be mirrored into this sub-directory for update.

\subsubsection{download}
This directory is used for transfering files from the \ovfc to the USB Stick. There is a subdirectory for each application transfering files (e.g. xcsoar).

\subsubsection{upload}
This directory is used for transfering file from the USB Stick to the \ovfc.
There is a subdirectory for different applications:
xcsoar: XCSoar home directory

\subsubsection{images}
This directory holds Software Image Files used for recovery of the whole \ovfc. 

\subsection{Download files for the USB Stick}

All files need for updating the \ovfc can be downloaded at

\begin{quote}
	\url{ftp://ftp.openvario.org/}
\end{quote}


\subsection{Recovering using USB Stick}
To recover the whole \ovfc a special file has to be stored in the \textbf{openvario} sub directory.

The file is named 

\hspace{15mm}\textbf{ov-recovery.itb}

and includes a complete linux kernel as well as a initramfs to bootup the \ovfc.
A user menu shows up after booting which enables the user to write a new image file on the SD Card.

The image file which should be written to SD Card has to reside in the 

\hspace{15mm}\textbf{images}

subdirectory. The file used is exactly the same as used for writting to a SD Card using a PC.




